 \documentclass[11pt]{article}

 \usepackage{amsmath}
 \usepackage{amsxtra}
 \usepackage{amsfonts}
 \usepackage{amssymb}
 \usepackage{amsthm}
 \usepackage{graphicx}
 \usepackage{mathtools}
 %\usepackage{dsfont}
 \usepackage{hyperref}

 \graphicspath{ {/Users/andrew/hw/math191k/figures/}}

 \usepackage[margin=3cm]{geometry}
 \newcommand{\eps}{\varepsilon}
 \newcommand{\Z}{\mathbb{Z}}
 \newcommand{\N}{\mathbb{N}}
 \newcommand{\R}{\mathbb{R}}
 \newcommand{\Q}{\mathbb{Q}}
 \newcommand{\W}{\mathcal{W}}
 \newcommand{\U}{\mathcal{U}}
 \newcommand{\V}{\mathcal{V}}
 \newcommand{\C}{\mathbb{C}}
 \newcommand{\K}{\mathcal{K}}

 \newcommand{\set}[1]{\{ #1 \}}
 \newcommand{\gen}[1]{\left\langle #1 \right\rangle}
 \newcommand{\floor}[1]{\left\lfloor #1 \right\rfloor}
 \newcommand{\abs}[1]{\left| #1 \right|}
 \renewcommand{\phi}{\varphi}
 \renewcommand{\Re}{\operatorname{Re}}
 \renewcommand{\Im}{\operatorname{Im}}
 \newcommand{\conj}[1]{\mkern 1.5mu\overline{\mkern-1.5mu#1\mkern-1.5mu}\mkern 1.5mu}
 \newcommand{\defeq}{\vcentcolon=}
 \newcommand{\identity}{\mathds{1}}

 \DeclareMathOperator{\im}{im}
 \DeclareMathOperator{\res}{Res}
 \DeclareMathOperator{\Arg}{Arg}
 \DeclareMathOperator{\parg}{arg}
 \DeclareMathOperator{\Log}{Log}

 \theoremstyle{plain}
 \newtheorem{thm}{Theorem}

 \theoremstyle{definition}
 \newtheorem{recall}{Recall}
 \newtheorem{remark}{Remark}
 \newtheorem{definition}{Definition}
 \newtheorem{prop}{Proposition}
 \newtheorem{cor}{Corollary}
 \newtheorem{lemma}{Lemma}
 \newtheorem{ex}{Example}
 \newtheorem{claim}{Claim}
 \newtheorem{exercise}{Exercise}
 \newtheorem{notation}{Notation}
 \newtheorem{question}{Question}


 \title{Dynamical systems theory $\bigcap$ Knot theory}

 \author{Andrew Berger\\ Yitz Deng}


\begin{document}
\maketitle

\tableofcontents

%%start here

\section{Introduction and background}

Initial exposition on the connection between dynamical systems theory and knot theory. The set of links.

\subsection{Templates}

Motivate: collapse out along stable manifolds


\begin{definition}
  Given a template $\tau$, we define
  \begin{itemize}
    \item Branch lines  $\set{l_j : j = 1, \cdots M}$
    \item Strips $\set{x_i : i = 1, \cdots N \geq 2 M}$
    \item Itinerary space $\Sigma_\tau = \set{a_0 a_1 a_2 \cdots} \subset \set{x_1, x_2, \ldots, x_N}^{\Z^+}$
    \item Transition matrix

      \begin{equation}
        A_\tau(i,j) = \left.\{ \
        \begin{smallmatrix}
          0: \not\exists \text{ a strip from } x_i \text{ to } x_j \\
          1: \exists \text{ a strip from } x_i \text{ to } x_j
        \end{smallmatrix}
      \end{equation}
  \end{itemize}
\end{definition}

Note that powers of $A_\tau$ capture allowable trajectories. (Can we expand this into saying anything interesting?)

Put short exposition on ordering $\vartriangleright$ of orbits


\subsection{Template renormalization}
Template renormalization exposition goes here. This should be done very carefully and in much greater detail than given in Ghrist, I think

\subsection{Mexico}
(What happens when you leave your computer unattended\dots)
mad max
hardcore henry
klean kanteen katie
burrito berger

\subsection{Braids}

\subsection{Link of knots}

\subsection{Renormalization}

\section{Outline of proof that $\mathcal{V}$ is a universal template}

\subsection{Braids and the theorem of Alexander}

Every link is isotopic to some closed braid on $P$ strands for some $P$. Cite the proper paper

\subsection{The templates $\W_q$}

An isotopic copy of any closed braid exists as a set of periodic oribts on some $W_q$ for sufficently large $q$.

\subsection{Construct $\W_{q+1}$ from $\W_q$}

Start with renormalized $\W_1 \in \V$, and append a pair of ears to\dots

\subsection{Find $\W_q \subset \V$ for all $q$}

This is very tricky and will have to be very loosely sketched




\section{Miscelaneous discussions and interesting open questions}




\end{document}
