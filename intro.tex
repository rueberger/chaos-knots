\documentclass[report.tex]{subfiles}

\section{Introduction and background}

This report is heavily based on Robert Ghrist's \emph{An ODE whose solutions contain all knots and links} \cite{knottyode}. In that manuscript, Ghrist gives an excellent high-level exposition of a proof that an ODE containing
all knots and links as periodic orbits exists, building heavily off of his existing work in this area. His manuscript is chock full of interesting side notes; the connection between knot theory and dynamical systems is so rich
that it's difficult to not get distracted by the

Rather than pointlessly recapitulate what Ghrist has already stated so elegantly, we seek to give a more compact and self-contained exposition of the beautiful proof that the required ODE exists. The reader is encouraged to
refer to Ghrist's review papers \cite{knottyode}\cite{chaoticknots}to see many many interesting footnotes and connections that will not make it into this work.

We will inevitably be unable to resist the temptation to discuss at least a few interesting connections and questions. The last section of this manuscript is reserved for discussion of those distractions that most fascinate the
authors during the writing.

As briefly stated above, our ultimate goal is to prove the existence of a universal ODE, an ODE whose periodic solutions contain all knots and links.


\subsection{Templates}

The story begins with \emph{templates}, a beautiful and very powerful construction. It would be very difficult to prove the existence of a universal ODE

Motivate: collapse out along stable manifolds


\begin{definition}
  Given a template $\tau$, we define
  \begin{itemize}
    \item Branch lines  $\set{l_j : j = 1, \cdots M}$
    \item Strips $\set{x_i : i = 1, \cdots N \geq 2 M}$
    \item Itinerary space $\Sigma_\tau = \set{a_0 a_1 a_2 \cdots} \subset \set{x_1, x_2, \ldots, x_N}^{\Z^+}$
    \item Transition matrix

      \begin{equation}
        A_\tau(i,j) = \left.\{ \
        \begin{smallmatrix}
          0: \not\exists \text{ a strip from } x_i \text{ to } x_j \\
          1: \exists \text{ a strip from } x_i \text{ to } x_j
        \end{smallmatrix}
      \end{equation}
  \end{itemize}
\end{definition}

Note that powers of $A_\tau$ capture allowable trajectories. (Can we expand this into saying anything interesting?)

Put short exposition on ordering $\vartriangleright$ of orbits


\subsection{Template renormalization}
Template renormalization exposition goes here. This should be done very carefully and in much greater detail than given in Ghrist, I think

\subsection{Braids}

\subsection{Link of knots}

\subsection{Renormalization}
